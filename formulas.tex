\title{Formulas}
\author{ 
        Philip D. Kent
}
\documentclass[12pt]{report}
\usepackage[utf8]{inputenc}
\usepackage[english]{babel}
\usepackage{physics}
\usepackage{amsmath,mathtools,amssymb}
\usepackage[]{hyperref}
\hypersetup{
    pdftitle={Formulas},
    pdfauthor={Philip D. Kent},
    pdfsubject={Physics},
    pdfkeywords={physics, mathematics},
    bookmarksnumbered=true,     
    bookmarksopen=true,         
    bookmarksopenlevel=1,       
    colorlinks=true,            
    pdfstartview=Fit,           
    pdfpagemode=UseOutlines,    
    pdfpagelayout=TwoPageRight
}
\usepackage{amsthm}

\newtheorem{theorem}{Theorem}[section]
\newtheorem{corollary}{Corollary}[theorem]
\newtheorem{lemma}[theorem]{Lemma}
\theoremstyle{definition}
\newtheorem{definition}{Definition}[section]
\setlength\parindent{0pt}

% commands
\newcommand{\ppmat}[1]{\begin{pmatrix}#1\end{pmatrix}} 
\newcommand{\vvmat}[1]{\begin{vmatrix}#1\end{vmatrix}} 
\newcommand{\haf}[1]{\frac{#1}{2}}
\newcommand{\eq}[1]{\begin{equation}#1\end{equation}}
\newcommand{\fr}[1]{\begin{flushright}#1\end{flushright}}
\newcommand{\frss}[1]{\begin{flushright}\scriptsize{#1}\end{flushright}}
\newcommand{\bs}[1]{\boldsymbol{#1}}
\begin{document}
\maketitle

\part{Mathematics}
\chapter{Vector Calculus}
\begin{equation}
	\vec A \cdot \vec B = \vec B \cdot \vec A
\end{equation}

\begin{equation}
	\vec A \cdot \vec B  = A B cos \theta
\end{equation}

\begin{equation}
	\vec A \times \vec B = -\vec B \times \vec A
\end{equation}

\section{Triple Products}
\begin{equation}
	\vec A \cdot (\vec B \times \vec C) = \vec B \cdot (\vec C \times \vec A) = \vec C \cdot (\vec A \times \vec B)
\end{equation}

\begin{equation}
	\vec A \cdot (\vec B \times \vec C) = \begin{vmatrix} 
	A_x&A_y&A_z\\
	B_x&B_y&B_z\\
	C_x&C_y&C_z 
	\end{vmatrix}
\end{equation}

\begin{equation}
	\vec A \cdot (\vec B \times \vec C) = (\vec A \times \vec B) \cdot C
\end{equation}

\begin{equation}
	\vec A \times (\vec B \times \vec C) = \vec B(\vec A \cdot \vec C) - \vec C(\vec A \cdot \vec B)
\end{equation}

\section{Derivatives}
\begin{equation}
	\nabla  =  \hat{x} \frac{\partial }{\partial x} +  \hat{y} \frac{\partial }{\partial y} + \hat{z} \frac{\partial }{\partial z} 
\end{equation}

\begin{equation}
	\nabla T = \frac{\partial T}{\partial x} \hat{x} + \frac{\partial T}{\partial y} \hat{y} + \frac{\partial T}{\partial z} \hat{z}
\end{equation}

\begin{equation}
	\nabla \cdot \vec v =  \frac{\partial v_x}{\partial x} + \frac{\partial v_y}{\partial y} + \frac{\partial v_z}{\partial z} 
\end{equation}

\begin{equation}
	\nabla \times \vec v = 
	\hat x \left( \frac{\partial v_z}{\partial y} - \frac{\partial v_y}{\partial z} \right) 
	+ \hat y \left( \frac{\partial v_x}{\partial z} - \frac{\partial v_z}{\partial x} \right) 
	+ \hat z \left( \frac{\partial v_y}{\partial x} - \frac{\partial v_x}{\partial y} \right)
\end{equation}

\begin{equation}
	\nabla'^2 \left(\frac{1}{|\vec{r} - \vec{r'}|}\right) = -4\pi \delta(\vec{r} - \vec{r'})
\end{equation}

\section{Product Rules}

\section{Cylindrical Coordinates}

\begin{equation}
	\nabla^2 f = \frac{1}{\rho}\frac{\partial}{\partial \rho}\left( \rho\frac{\partial f}{\partial \rho} \right)
	+ \frac{1}{\rho^2}\frac{\partial^2 f}{\partial \phi^2} 
	+ \frac{\partial^2 f}{\partial z^2}
\end{equation}

\section{Spherical Coordinates}

\begin{equation}
	\nabla^2 f = \frac{1}{r^2}\frac{\partial }{\partial r}\left( r^2 \frac{\partial f}{\partial r} \right)
	+ \frac{1}{r^2 sin \theta}\frac{\partial}{\partial \theta}\left(sin \theta \frac{\partial f}{\partial \theta}\right)
	+\frac{1}{r^2 sin^2 \theta}\frac{\partial^2 f}{\partial \phi^2}
\end{equation}

\section{Green Functions}
\subsection{Green's Identities}
Green's first identity:
\begin{equation}
	\int_v \left( \phi \nabla'^2\psi + \nabla' \phi \cdot \nabla' \psi \right)d^3x' = \oint_s \phi \frac{\partial \psi}{\partial n} da'
\end{equation}
Green's second identity or Green's Theorem:
\begin{equation}
	\int_{v} \left(\phi \nabla'^2\psi - \psi \nabla'^2\phi \right)d^3x 
	= \oint_s \left[ \phi \frac{\partial \psi}{\partial n} - \psi \frac{\partial \phi}{\partial n} \right]da'  
\end{equation}

\subsection{Green Functions}
\begin{equation}
	\nabla'^2G(\vec{x}, \vec{x'}) = -4\pi \delta(\vec{x} - \vec{x'})
\end{equation}
\begin{equation}
	G(\vec{x}, \vec{x'}) = \frac{1}{|\vec{x} - \vec{x'}|} + F(\vec{x}, \vec{x'})
\end{equation}
\begin{equation}
	\nabla'^2 F(\vec{x}, \vec{x'}) = 0
\end{equation}


\subsection{Boundary Conditions}
Dirichlet:
\begin{equation}
	\begin{split}
		\Phi(\vec x)  \text{   specified on S} \\
		G_D(\vec x, \vec x') = 0 \text{   for $\vec x'$ on S}
	\end{split}
\end{equation}
Neumann:
\begin{equation}
	\begin{split}
		\frac{\partial \Phi}{\partial n} \text{   specified on S} \\
		\frac{\partial G_N}{\partial n'}(\vec x, \vec x') = -\frac{4 \pi}{S} \text{   for $\vec x'$ on S}
	\end{split}
\end{equation}
\chapter{Linear Algebra}


\section{Determinant Properties}
\begin{theorem}
	The determinant of a matrix A is non-zero if and only if A is invertible 
\end{theorem}
\eq{\det(A^{-1})=\frac{1}{\det(A)}}
\eq{\det(A)=\prod_{i=1}^n \lambda_i}
\frss{Determinant is product of eigenvalues}
\eq{\det(A B) = \det(A)\det(B)}

\section{Orthogonal Matrices}
\begin{definition}
	An nxn real matrix A is orthogonal if the columns of A are orthonormal.
\end{definition}
\begin{theorem}
	Every orthogonal matrix is invertible.
\end{theorem}
\begin{theorem}
	The inverse of an orthogonal matrix orthogonal.
\end{theorem}
\eq{A^TA=I}
\eq{A^T=A^{-1}}
\eq{\det(A) = \pm 1}

\section{Transpose Properties}
\eq{
	\left(\bs{A}+\bs{B}\right)^T=\bs{A}^T+\bs{B}^T
}
\eq{
	\left(\boldsymbol{A}\boldsymbol{B}\right)^T=\boldsymbol{B}^T\boldsymbol{A}^T
}
\eq{
	\det\left(\bs{A}^T\right) = \det(\bs{A})
}

\section{Linear Vector Spaces}
\begin{definition}{Linear Vector Space.}
A linear vector space V is a collection of objects $\ket{1}, \ket{2},...,\ket{V},...\ket{W},...,$ called vectors, for which there exists \\

\begin{itemize}
	\item A definite rule for forming the vector sum, denoted $\ket{V} + \ket{W}$
	\item A definite rule for multiplication by scalars $a,b,...,$ denoted $a\ket{V}$ with the following features:
	\begin{enumerate}
		\item Closure under addition: $\ket{V} + \ket{W} \in V$
		\item Scalar multiplication is distributive in the vectors: $a(\ket{V} + \ket{W}) = a\ket{V} + a\ket{W}$
		\item Scalar multiplication is distributive in the scalars: $(a + b)\ket{V} = a\ket{V} + b\ket{V}$
		\item Scalar multiplication is associative $a(b\ket{V}) = ab\ket{V}$
		\item Addition is commutative: $\ket{V} + \ket{W} = \ket{W} + \ket{V}$
		\item Addition is associative: $\ket{V} + (\ket{W} + \ket{Z}) = (\ket{V} + \ket{W}) + \ket{Z}$
		\item There exists a null vector $\ket{0}$ obeying $\ket{V} + \ket{0} = \ket{V}$
		\item For every vector there exists an inverse under addition, $\ket{-V}$, such that $\ket{V} + \ket{-V} = \ket{0}$
	\end{enumerate}
\end{itemize}
\end{definition}

\begin{definition}{Field.} 
The numbers $a,b,...$ are called the field over which the vector space is defined.
\end{definition}

\begin{theorem}{}
	The above axioms of vector spaces imply:
	\begin{enumerate}
		\item $\ket{0}$ is unique
		\item $0\ket{V} = \ket{0}$
		\item $\ket{-V} = -\ket{V}$
		\item $\ket{-V}$ is the unique additive inverse of $\ket{V}$
	\end{enumerate}
\end{theorem}

\subsection{Linear Independence}
\begin{definition}{Linearly Independent.}
	A set of vectors $\{\ket{i}\}$ is said to be linearly independent if $\sum_{i=1}^{n}a_i\ket{i} = \ket{0}$ only when all $a_i=0$.
\end{definition}

\begin{definition}{Dimension.}
	A vector space has a dimension $n$ if it can accommodate a maximum of $n$ linearly independent vectors.
\end{definition}

\begin{theorem}
	Any vector $\ket{V}$ in an $n$-dimensional space can be written as a linear combination of $n$ linearly independent vectors $\ket{1}...\ket{n}$.
\end{theorem}

\begin{theorem}
	The vector expansion $\ket{V} = \sum_{i}^nv_i\ket{i}$ is unique.
\end{theorem}

\subsection{Inner Product Space}
\begin{definition}{Inner Product Space.}
	A vector space with an inner product is called an inner product space where the inner product has the following properties:
	\begin{itemize}
		\item $\braket{V}{W}=\braket{W}{V}^*$ (skew-symmetric)
		\item $\braket{V}{V} \geq 0$ and $\braket{V}{V} = 0$ iff $\ket{V}=\ket{0}$ (positive semidefiniteness)
		\item $\braket{V}{(a\ket{W} + b\ket{z})} = \braket{V}{aW + bZ} = a\braket{V}{W} + b\braket{V}{Z}$ (linearity in ket)
	\end{itemize}
\end{definition}

\subsection{Basis of a Vector Space}
\begin{definition}{Basis.}
	A set of $n$ linearly independent vectors in an $n$-dimensional space is called a basis.
\end{definition}

\begin{definition}{Components of a Vector.}
	The coefficients of expansion $v_i$ of a vector in terms of a linearly independent basis $\{\ket{i}\}$ are called the components of the vector in that basis.
\end{definition}

\begin{equation}
	\braket{i}{j} = \delta_{ij} \text{ for an orthonormal basis}
\end{equation}

\begin{equation}
	\sum_{i=1}^n \ket{i}\bra{i} = I \text{ for a complete basis}
\end{equation}

\section{Dual Spaces}

\subsection{Projection Operator}
$\ket{i}\bra{i}$ is the projection operator that projects onto the vector $\ket{i}$
\begin{equation}
	\ket{V} = \sum_{i} \ket{i}\braket{i}{V}
\end{equation}	

The adjoint of the vector $\ket{V} = \sum_i v_i \ket{i}$ is:
\begin{equation}
	\bra{V} = \sum_i \bra{i} v_i^* 
\end{equation}

The adjoint of the vector $\ket{V} = \sum_i \ket{i}\braket{i}{V}$ is:
\begin{equation}
	\bra{V} = \sum_i \braket{V}{i}\bra{i}
\end{equation}


\subsection{Schwarz and Triangle Inequality}
\begin{equation}
	|\braket{V}{W}| \leq |V||W| \text{  Schwarz Inequality}
\end{equation}

\begin{equation}
	|V + W| \leq |V| + |W| \text{  Triangle Inequality}
\end{equation}

\section{Matrix Inverse}
\subsection{2x2 Matrix Inverse}
 \eq{
 	A^{-1} = \ppmat{a && b \\ c && d}^{-1} = \frac{1}{det(A)}\ppmat{d && -b \\ -c && a} = \frac{1}{ad - bc}\ppmat{d && -b \\ -c && a}
 }
\subsection{3x3 Matrix Inverse}
\eq{
	A^{-1} = \ppmat{a&&d&&g\\b&&e&&h\\c&&f&&i}^{-1} = \frac{1}{\text{det}(A)}\ppmat{
			\vvmat{e&&h\\f&&i}&&\vvmat{g&&d\\i&&f}&&\vvmat{d&&g\\e&&h}\\[16pt]
			\vvmat{h&&b\\i&&c}&&\vvmat{a&&g\\c&&i}&&\vvmat{g&&a\\h&&b}\\[16pt]
			\vvmat{b&&e\\c&&f}&&\vvmat{d&&a\\f&&c}&&\vvmat{a&&d\\b&&e}
		}
}

\section{Operators}

\subsection{Commutators}
\begin{equation}
	[\Omega, \Lambda\theta] = \Lambda[\Omega, \theta] + [\Omega, \Lambda]\theta
\end{equation}
\begin{equation}
	[\Lambda\Omega, \theta] = \Lambda[\Omega, \theta] + [\Lambda, \theta]\Omega
\end{equation}

\subsection{Inverses}
The inverse of a product of operators is the product of the inverses in reverse:
\begin{equation}
	(\Omega\Lambda)^{-1} = \Lambda^{-1}\Omega^{-1}
\end{equation}


\subsection{Matrix Elements of an Operator}
\begin{equation}
	\braket{i}{j'} = \bra{i}\Omega\ket{j} = \Omega_{ij}
\end{equation}
The matrix element $\Omega_{ij}$ is the $i^{th}$ component of the $j^{th}$ basis vector after it has been transformed by $\Omega$.

\subsection{Active and Passive Transformations}
Under a change of basis by the unitary operator $U$
\begin{align}
	\ket{V} &\rightarrow U\ket{V} \text{  (active)}\\
	\Omega  &\rightarrow U^{\dagger} \Omega U \text{  (passive)}
\end{align}

\subsection{Trace of an Operator}

\begin{equation}
	Tr (A) = \sum_i A_{ij}
\end{equation}

Properties: permutations are cyclic, the trace of an operator is independent of basis
\begin{align}
	Tr(AB) &= Tr(BA) \\
	Tr(ABC) &= Tr(BCA) = Tr(CAB) \\
	Tr(A) &= Tr(U^\dagger A U)
\end{align}

\subsection{Hermitian Operators}

\begin{definition}{Hermitian Operator:}
	An operator $\Omega$ is Hermitian if $\Omega^\dagger = \Omega$
\end{definition}

\begin{definition}{Anti-Hermitian Operator:}
	An operator $\Omega$ is anti-Hermitian if $\Omega^\dagger=-\Omega$
\end{definition}

Hermitian operators are like real numbers, anti-Hermitian operators are like complex numbers.  Any operator can be decomposed into Hermitian and 
anti-Hermitian parts:

\begin{equation}
	\Omega = \frac{\Omega+\Omega^\dagger}{2} + \frac{\Omega-\Omega^\dagger}{2}
\end{equation}

\begin{theorem}
	The eigenvalues of a Hermitian operator are real valued.
\end{theorem}

\begin{theorem}
	To every Hermitian operator $\Omega$, there exists (at least) a basis consisting of its orthonormal eigenvectors.  The operator
	is diagonal in this eigenbasis and has its eigenvalues as its diagonal entries.
\end{theorem}

\begin{theorem}
	The eigenvectors of a hermitian operator belonging to distinct eigenvalues are orthogonal.
\end{theorem}

\begin{theorem}
	The eigen vectors of a hermitian operator span the space.
\end{theorem}

\subsection{Unitary Operators}
\begin{definition}{Unitary Operator:}
	An operator $U$ is unitary if \\ $UU^\dagger=I$
\end{definition}
\begin{theorem}
	The columns of a unitary matrix are orthonormal and the rows of a unitary matrix are orthonormal as well.
\end{theorem}
\begin{theorem}
	The eigenvalues of a unitary operator are complex numbers of unit modulus.
\end{theorem}

\begin{theorem}
	The eigenvectors of a unitary operator are mutually orthogonal.
\end{theorem}

\subsection{Diagonalization of Hermitian Matrices}
If $\Omega$ is a Hermitian matrix, there exists a unitary matrix $U$ (built out of the eigenvectors of $\Omega$) such that 
\begin{equation}
	U^\dagger\Omega U
\end{equation}
is diagonal.

\begin{theorem}
	if $\Omega$ and $\Lambda$ are two commuting Hermitian operators, there exists (at least) a basis of common eigenvectors that 
	diagonalizes them both.  The common eigenbasis is unique if only one of $\Omega$ or $\Lambda$ are degenerate in some subspace.  
	If both $\Omega$ and $\Lambda$ are	degenerate in some subspace, the common eigenbasis is not unique.
\end{theorem}

\begin{theorem}
	One can always find, for a finite n, a set of operators $\{\Omega, \Lambda, \Gamma, ...\}$ that commute with each other and define
	a unique eigenbasis that is shared by all operators in the set.
\end{theorem}

\begin{definition}{Complete set of commuting operators:}
	A set of commuting operators $\{\Omega, \Lambda, \Gamma, ...\}$ that share a unique eigenbasis.
\end{definition}




\section{Eigenvectors and Eigenvalues}

Each operator has eigen vectors associated with it:
\begin{equation}
	\Omega\ket{V} = \omega \ket{V}
\end{equation}
\begin{equation}
	(\Omega - \omega I)\ket{V} = \ket{0}
\end{equation}
\begin{equation}
	\sum_j (\Omega_{ij}-\omega\delta_{ij})v_j = 0
\end{equation}
which gives the eigenvectors if the eigenvalue is known.

$(\Omega - \omega I)^{-1}$ does not exist so the eigenvalue problem cannot be directly solved:
\begin{equation}
	\ket{V} = (\Omega - \omega I)^{-1}\ket{0} = \ket{0}
\end{equation}

\subsection{Characteristic Equation and Polynomial}
Characteristic equation
\begin{equation}
	\text{det}(\Omega - \omega I) = 0
\end{equation}
gives the eigenvalues.
\\
The characteristic equation is a polynomial equation containing the characteristic polynomial:
\begin{equation}
	P^n(\omega) = \sum_{m=0}^{n} c_m \omega^m = 0
\end{equation}

Roots of the characteristic polynomial are the eigenvalues.

\subsection{Theorems}
\begin{theorem}
	The eigenvalues of an operator $\Omega$ are covariant, they have the same value in any basis.
\end{theorem}

\begin{theorem}
	A Hermitian or Unitary operator in $V^n(C)$ has n eigenvalues.
\end{theorem}

\section{Functions of Operators}

\begin{definition}{c-number (or classical number):}
	Refers real or complex numbers which are commuting.
\end{definition}

\begin{definition}{q-number(or quantum number):}
	Refers to operators which in general do not commute.
\end{definition}
If only one q-number is present in an equation, everything commutes and it can be treated as a c-number. 
\\
If $\Omega$ is an operator:
\begin{equation}
	e^\Omega = \sum_{n=1}^{\infty} \frac{\Omega^n}{n!}
\end{equation}	


If $\Omega$ is Hermitian, then in its eigenbasis:
\begin{equation}
	e^\Omega = \begin{bmatrix}
		\sum_{m=0}^{\infty} \frac{\omega_1^m}{m!} && && \\
		&& \ddots && \\
		&& && \sum_{m=0}^{\infty} \frac{\omega_n^m}{m!}
	\end{bmatrix}
\end{equation}


If $\theta(\lambda)$ is the operator:
\begin{equation}
	\theta(\lambda) = e^{\lambda \Omega}
\end{equation}

\begin{equation}
	\frac{d\theta (\lambda)}{d\lambda} = \Omega e^{\lambda \Omega} = \theta(\lambda)\Omega
\end{equation}
\begin{equation}
	\theta (\lambda) = c exp\left( \int_0^{\lambda} \Omega d\lambda' \right) = c exp(\Omega \lambda)
\end{equation}

\section{Infinite Dimensional Vector Spaces}

\begin{equation}
	\braket{f}{g} = \int_a^b f^*(x)g(x)dx \text{ inner product}
\end{equation}

\begin{equation}
	\braket{x}{x'} = \delta(x-x') \text{ normalization condition of basis vectors}
\end{equation}

\section{Derivative Operator}
\begin{equation}
	D\ket{f} = \ket{df/dx}
\end{equation}
\begin{equation}
	\bra{x}D\ket{x'} = D_{xx'} = \delta'(x-x') = \delta(x-x')\frac{d}{dx'}
\end{equation}

The derivative operator is not hermitian.  $K=-iD$ is Hermitian in the space of functions obeying:

\begin{equation}
	-ig^*(x)f(x)\bigg\rvert_a^b = 0
\end{equation}

\chapter{Lie Algebra}

\eq{
	X_i=-i\frac{\partial M(\vec{\theta})}{\partial \theta_i}\bigg\rvert_{\vec{\theta}=0} \text{   (Generators)}
}
\frss{The generators contain all information about the group. }

\eq{
	M^{gen}({\theta}_1, {\theta}_2, ...) = e^{i\theta^i X_i}
}
\frss{The generators give the group elements}
\eq{
	\left[X_i, X_j\right] = i C_{i,j}^k X_k \text{   (Lie Algebra)}
}
\frss{Must be true for $M^{gen}$ to be a valid representation of the group \\
$C_{i,j}^k$ (structure constants) are independent of the group representation \\
This is for passive transformations, for active transformations $i\rightarrow-i$ }

Notes:
\begin{enumerate}
\item The representations of the generators and the group elements belong to different spaces and behave differently
\item The Lie Algebra is not a group
\end{enumerate}

\section{SU(2)}

\eq{
h = \ppmat{a && c \\ b && d} \newline \{a, b, c, d\} \in \mathbb{C}
}
\frss{ 8 parameters}

\eq{
h^{\dagger}h = I \implies h^\dagger = h^{-1}
}
\frss{Four equations so now 4 parameters}

\eq{
h = \ppmat{a && -b^* \\ b && a^* }
}

\frss{Proof: $h^{-1} = \ppmat{d && -c \\ -b && a} = h^\dagger =  \ppmat{a^* && b^* \\ c^* && d^*}$}
\frss{(Four equations above are $d=a^*$, $c=-b^*$)}

\eq{
\text{det}(h) = 1
}

\eq{
	\abs{a}^2 + \abs{b}^2 = 1
}
\frss{One constraint so now 3 parameters ($\boldsymbol{R^3}$)}
Only 3 free parameters so SU(2) is a three dimensional group.

\eq{
	\boxed{\text{SU}(2) = \left\{ \ppmat{a && -b^* \\ b && a^* } : a, b \in \mathbb{C}, \abs{a}^2+\abs{b}^2=1   \right\}}
}
Parameterized by $\Im(a), \Re(b), \Im(b)$

\subsection{Generators of SU(2)}
\eq{
	X_1=\ppmat{0 && 1 \\ 1 && 0}=\sigma_1\text{, }X_2=\ppmat{0&&-i\\i&&0}=\sigma_2\text{, }X_3=\ppmat{1&&0\\0&&-1}=\sigma_3
}
\frss{These are the Pauli Matrices}

\subsection{Lie Algebra of SU(2)}
\eq{\left[X_i, X_y\right]=i\epsilon_{ijk}X_k}
\frss{Same algebra as SO(3)}

\subsection{Topology}
If $a = x + iy$, $b = z + ir$ then
\eq{
	\text{det}(h) = x^2+y^2+z^2+r^2 = 1  
}
and the topology is $S^3$, the unit 3-sphere in $\boldsymbol{R^4}$

\subsection{Lie Algebra}
\eq{
	\left[X_i,X_j\right]=i\epsilon_{ijk}X_k
}

\subsection{Basis}
Basis of SU(2) is given by
\eq{
	u_1=\ppmat{0 && i \\ i && 0} \text{     } u_2=\ppmat{0 && -1 \\ 1 && 0} \text{     } u_3=\ppmat{i && 0 \\ 0 && -i}
}

\section{SO(3)}
\eq{
	h = \ppmat{a&&d&&g\\b&&e&&h\\c&&f&&i}
}
\frss{9 parameters}
\eq{
	\text{det}(h) = 1
}
\frss{3 parameters}
\eq{
	h^T=h^{-1}
}
\frss{6 parameters}


Representation $M(\theta_1, \theta_2, \theta_3) = M_1(\theta_1)M_2(\theta_2)M_3(\theta_3)$

\scriptsize{
\eq{
	M_1=\ppmat{1&&0&&0\\0&&cos(\theta_1)&&-sin(\theta_1)\\0&&sin(\theta_1)&&cos(\theta_1)}
	\text{ }
	M_2=\ppmat{cos(\theta_2)&&0&&-sin(\theta_2)\\0&&1&&0\\sin(\theta_2)&&0&&cos(\theta_2)}
	\text{ }
	M_3=\ppmat{cos(\theta_3)&&-sin(\theta_3)&&0\\sin(\theta_3)&&cos(\theta_3)&&0\\0&&0&&1}
}}
Generators:
\eq{
	X_1 = -i\ppmat{0&&0&&0\\0&&0&&-1\\0&&1&&0}\text{ }X_2=-i\ppmat{0&&0&&-1\\0&&0&&0\\1&&0&&0}\text{ }X_3=-i\ppmat{0&&-1&&0\\1&&0&&0\\0&&0&&0}
}
\subsection{Lie Algebra}
Structure constants $C_{ij}^k$ are the Levi-Civita tensor: 
\eq{
	[X_i,X_j]=i\epsilon^`{ijk}X_k
}
\eq{
	[X_1,X_2]=iX_3\text{, }[X_2,X_3]=iX_1\text{, }[X_3,X_1]=iX_2
}

\section{SO(1, 3)}

\eq{h^\dagger g h = g }
\frss{Generalized orthogonality condition, g is the metric}
\chapter{Dirac Delta Function}

The dirac delta function is even
\begin{equation}
	\delta(x-x') = \delta(x'-x)
\end{equation}

Derivative of the dirac delta function:
\begin{equation}
	\delta'(x-x') = \frac{d}{dx}\delta(x-x') = - \frac{d}{dx'}\delta(x-x')
\end{equation}

\begin{equation}
	\int \delta'(x-x')f(x')dx' = \frac{d}{dx} f(x)
\end{equation}

$\delta'(x-x')$ is an odd function

\begin{equation}
	\frac{d^n\delta(x-x')}{dx^n} = \delta(x-x')\frac{d^n}{dx'^n}
\end{equation}

Definition of the dirac delta function:
\begin{equation}
	\delta(x-x') = \frac{1}{2\pi} \int_{-\infty}^{\infty} e^{ik(x-x')}dk
\end{equation}
\chapter{Fourier Analysis}

\begin{equation}
	f(k) = \frac{1}{\sqrt{2\pi}}\int_{-\infty}^{\infty}	e^{-ikx}f(x)dx
\end{equation}
\begin{equation}
	f(x) = \frac{1}{\sqrt{2\pi}}\int_{-\infty}^{\infty} e^{ikx} f(k) dk
\end{equation}


\part{Physics}
\chapter{Electrodynamics}

\section{Poisson Equation}

\begin{equation}
	\nabla^2 \Phi = -\rho/\epsilon_0
\end{equation}
Integral form ($R=|\vec x - \vec x'|$):
\begin{equation}
	\Phi(\vec x) = \frac{1}{4 \pi \epsilon_0} \int_v \frac{\rho(\vec{x}') }{R} d^3x' + 
	\frac{1}{4\pi}\oint_s \left[ \frac{1}{R} \frac{\partial \Phi}{\partial n'} - \Phi \frac{\partial}{\partial n'}\left( \frac{1}{R} \right) \right] da'
\end{equation}

\subsection{Solutions to the Poisson Equation}
Dirichlet boundary conditions:
\begin{equation}
	\begin{split}
		G_D(\vec x, \vec{x}') &= 0  \text{  for $\vec{x}'$ on $S$} \\
		\Phi(\vec{x}) &= \frac{1}{4 \pi} \int_V \rho(\vec{x}')G_D(\vec{x}, \vec{x}')d^3x' - \frac{1}{4 \pi} \oint_S \Phi(\vec{x}')\frac{\partial G_D}{\partial n'}da' 	
	\end{split}
\end{equation}
Neumann boundary conditions:
\begin{equation}
	\begin{split}
		\frac{\partial G_N}{\partial n'}(\vec x, \vec{x}')  &= -\frac{4 \pi}{S}  \text{  for $\vec{x}'$ on $S$} \\
		\Phi(\vec{x}) &= \langle \Phi \rangle_S 
		+ \frac{1}{4 \pi \epsilon_0} \int_V \rho(\vec{x}')G_N(\vec{x}, \vec{x}')d^3x' 
		+ \frac{1}{4 \pi} \oint_S \frac{\partial \Phi}{\partial n'} G_N da' 	
	\end{split}
\end{equation}

\subsection{Potential Energy}
\begin{equation}
	W = \frac{1}{2} \int \rho(\vec{x}) \Phi(\vec{x}) d^3x	
\end{equation}
\begin{equation} \label{eq:edfieldpe}
	W = \frac{\epsilon_0}{2} \int \left| \vec{E} \right|^2 d^3x
\end{equation}
Notes:
\begin{enumerate}
	\item \ref{eq:edfieldpe} includes self energies
\end{enumerate}


\section{Special Relativity}

\subsection{Constant Acceleration}
$\tau$ is proper time, $x$ is coordinate distance, $t$ is coordinate time
\begin{equation}
	x(\tau) = \frac{c^2}{a} \left( cosh \frac{a \tau}{c} - 1 \right) 
\end{equation}
\begin{equation}
t(\tau) = \frac{c}{a} sinh \frac{a \tau}{c}
\end{equation}

\begin{equation}
	c = 3 \times 10^8 \text{  m/s}
\end{equation}
\begin{equation}
	\beta = \pi \times 10^7 \text{  s/yr}
\end{equation}

\begin{equation}
x_{lyr} = \frac{x_m}{c \beta}
\end{equation}
\begin{equation}
\tau_{yr} = \frac{\tau_{sec}}{\beta} 
\end{equation}

For 1g of acceleration, $a = g$

\begin{equation}
	x_{lyr}(\tau) = \frac{c^2}{g \beta c} cosh \left(\frac{g \beta \tau_{yr}}{c} - 1\right)
\end{equation}

\begin{equation}
	g\beta \approx c
\end{equation}

\begin{equation}
	x_{lyr}(\tau) \approx cosh \left( \tau_{yr} - 1\right)
\end{equation}

\begin{equation}
	t_{yr}(\tau_{yr}) \approx sinh \left( \tau_{yr} \right)
\end{equation}
\chapter{Quantum Mechanics}

\section{Theorems}
\begin{theorem}
	There is no degeneracy in on-dimensional bound states.
\end{theorem}

\begin{theorem}
	The eigenfunctions of a real valued Hamiltonian can always be chosen pure real in the coordinate basis.
\end{theorem}


\begin{theorem}
	The Hermiticity of the Hamiltonian is preserved under a unitary change of basis, the reality of the Hamiltonian is 
	not.
\end{theorem}

\section{Operators}

\begin{center}
\begin{tabular}{ |c|c|c|}
 \hline
   & Position Space & Momentum Space \\
  Position Operator & x & $i\hbar \frac{\partial}{\partial p}$ \\
  Momentum Operator & $-i\hbar \frac{\partial}{\partial x}$ & p \\
 \hline
\end{tabular}
\end{center}



\part{Misc}
\section{Eigenvalues and Eigenvectors of Rotation Matrix}
The rotation matrix 
\begin{center}
$
 \Omega = \begin{pmatrix}
 	cos(\theta) && sin(\theta) \\
 	-sin(\theta) && cos(\theta)
 \end{pmatrix}
$ 
\end{center}
has eigenvectors and eigenvalues 
\begin{center}
$
	\ket{\omega_1} = \frac{1}{\sqrt{2}}\ppmat{i\\1}
$, $\omega_1 = e^{-i\theta} $
and
$
	\ket{\omega_2} = \frac{1}{\sqrt{2}}\ppmat{-i\\1}
$, $ \omega_2 = e^{i\theta} $
\end{center}

A vector can be expressed as a linear combination of the eigenvectors

\begin{align*}
	\ket{\psi} &= c_1 \ket{\omega_1} + c_2 \ket{\omega_2} \\
	\ket{\psi} &= \braket{\omega_1}{\psi}\ket{\omega_1} + \braket{\omega_2}{\psi}\ket{\omega_1} \\
	\ket{\psi} &= \left( \ket{\omega_1}\bra{\omega_1} + \ket{\omega_2}\bra{\omega_2} \right)\ket{\psi} \\
	\ket{\omega_1}\bra{\omega_1} + \ket{\omega_2}\bra{\omega_2} &= I
\end{align*}

The image of $\ket{\psi}$ is

\begin{align*}
	\Omega \ket{\psi} &= \left( \Omega \ket{\omega_1}\bra{\omega_1} + \Omega \ket{\omega_2}\bra{\omega_2} \right)\ket{\psi} \\ 
 		&= \left( \omega_1 \ket{\omega_1}\bra{\omega_1} + \omega_2 \ket{\omega_2}\bra{\omega_2} \right)\ket{\psi} \\ 
\end{align*}

So

\begin{align*}
	\Aboxed{\Omega &= \omega_1 \ket{\omega_1}\bra{\omega_1} + \omega_2 \ket{\omega_2}\bra{\omega_2}} 
\end{align*}

For the rotation matrix

\begin{align*}
	\Omega&=\frac{e^{-i\theta}}{2}\ppmat{i\\1}\ppmat{-i&&1}+\frac{e^{i\theta}}{2}\ppmat{-i\\1}\ppmat{i&&1} \\
	\Aboxed{\Omega&=e^{-i\theta}\haf{1}\ppmat{1&&i\\-i&&1}+e^{i\theta}\haf{1}\ppmat{1&&-i\\i&&1}}
\end{align*}



\end{document}
